% Options for packages loaded elsewhere
% Options for packages loaded elsewhere
\PassOptionsToPackage{unicode}{hyperref}
\PassOptionsToPackage{hyphens}{url}
\PassOptionsToPackage{dvipsnames,svgnames,x11names}{xcolor}
%
\documentclass[
  letterpaper,
  DIV=11,
  numbers=noendperiod]{scrartcl}
\usepackage{xcolor}
\usepackage{amsmath,amssymb}
\setcounter{secnumdepth}{5}
\usepackage{iftex}
\ifPDFTeX
  \usepackage[T1]{fontenc}
  \usepackage[utf8]{inputenc}
  \usepackage{textcomp} % provide euro and other symbols
\else % if luatex or xetex
  \usepackage{unicode-math} % this also loads fontspec
  \defaultfontfeatures{Scale=MatchLowercase}
  \defaultfontfeatures[\rmfamily]{Ligatures=TeX,Scale=1}
\fi
\usepackage{lmodern}
\ifPDFTeX\else
  % xetex/luatex font selection
\fi
% Use upquote if available, for straight quotes in verbatim environments
\IfFileExists{upquote.sty}{\usepackage{upquote}}{}
\IfFileExists{microtype.sty}{% use microtype if available
  \usepackage[]{microtype}
  \UseMicrotypeSet[protrusion]{basicmath} % disable protrusion for tt fonts
}{}
\makeatletter
\@ifundefined{KOMAClassName}{% if non-KOMA class
  \IfFileExists{parskip.sty}{%
    \usepackage{parskip}
  }{% else
    \setlength{\parindent}{0pt}
    \setlength{\parskip}{6pt plus 2pt minus 1pt}}
}{% if KOMA class
  \KOMAoptions{parskip=half}}
\makeatother
% Make \paragraph and \subparagraph free-standing
\makeatletter
\ifx\paragraph\undefined\else
  \let\oldparagraph\paragraph
  \renewcommand{\paragraph}{
    \@ifstar
      \xxxParagraphStar
      \xxxParagraphNoStar
  }
  \newcommand{\xxxParagraphStar}[1]{\oldparagraph*{#1}\mbox{}}
  \newcommand{\xxxParagraphNoStar}[1]{\oldparagraph{#1}\mbox{}}
\fi
\ifx\subparagraph\undefined\else
  \let\oldsubparagraph\subparagraph
  \renewcommand{\subparagraph}{
    \@ifstar
      \xxxSubParagraphStar
      \xxxSubParagraphNoStar
  }
  \newcommand{\xxxSubParagraphStar}[1]{\oldsubparagraph*{#1}\mbox{}}
  \newcommand{\xxxSubParagraphNoStar}[1]{\oldsubparagraph{#1}\mbox{}}
\fi
\makeatother

\usepackage{color}
\usepackage{fancyvrb}
\newcommand{\VerbBar}{|}
\newcommand{\VERB}{\Verb[commandchars=\\\{\}]}
\DefineVerbatimEnvironment{Highlighting}{Verbatim}{commandchars=\\\{\}}
% Add ',fontsize=\small' for more characters per line
\usepackage{framed}
\definecolor{shadecolor}{RGB}{241,243,245}
\newenvironment{Shaded}{\begin{snugshade}}{\end{snugshade}}
\newcommand{\AlertTok}[1]{\textcolor[rgb]{0.68,0.00,0.00}{#1}}
\newcommand{\AnnotationTok}[1]{\textcolor[rgb]{0.37,0.37,0.37}{#1}}
\newcommand{\AttributeTok}[1]{\textcolor[rgb]{0.40,0.45,0.13}{#1}}
\newcommand{\BaseNTok}[1]{\textcolor[rgb]{0.68,0.00,0.00}{#1}}
\newcommand{\BuiltInTok}[1]{\textcolor[rgb]{0.00,0.23,0.31}{#1}}
\newcommand{\CharTok}[1]{\textcolor[rgb]{0.13,0.47,0.30}{#1}}
\newcommand{\CommentTok}[1]{\textcolor[rgb]{0.37,0.37,0.37}{#1}}
\newcommand{\CommentVarTok}[1]{\textcolor[rgb]{0.37,0.37,0.37}{\textit{#1}}}
\newcommand{\ConstantTok}[1]{\textcolor[rgb]{0.56,0.35,0.01}{#1}}
\newcommand{\ControlFlowTok}[1]{\textcolor[rgb]{0.00,0.23,0.31}{\textbf{#1}}}
\newcommand{\DataTypeTok}[1]{\textcolor[rgb]{0.68,0.00,0.00}{#1}}
\newcommand{\DecValTok}[1]{\textcolor[rgb]{0.68,0.00,0.00}{#1}}
\newcommand{\DocumentationTok}[1]{\textcolor[rgb]{0.37,0.37,0.37}{\textit{#1}}}
\newcommand{\ErrorTok}[1]{\textcolor[rgb]{0.68,0.00,0.00}{#1}}
\newcommand{\ExtensionTok}[1]{\textcolor[rgb]{0.00,0.23,0.31}{#1}}
\newcommand{\FloatTok}[1]{\textcolor[rgb]{0.68,0.00,0.00}{#1}}
\newcommand{\FunctionTok}[1]{\textcolor[rgb]{0.28,0.35,0.67}{#1}}
\newcommand{\ImportTok}[1]{\textcolor[rgb]{0.00,0.46,0.62}{#1}}
\newcommand{\InformationTok}[1]{\textcolor[rgb]{0.37,0.37,0.37}{#1}}
\newcommand{\KeywordTok}[1]{\textcolor[rgb]{0.00,0.23,0.31}{\textbf{#1}}}
\newcommand{\NormalTok}[1]{\textcolor[rgb]{0.00,0.23,0.31}{#1}}
\newcommand{\OperatorTok}[1]{\textcolor[rgb]{0.37,0.37,0.37}{#1}}
\newcommand{\OtherTok}[1]{\textcolor[rgb]{0.00,0.23,0.31}{#1}}
\newcommand{\PreprocessorTok}[1]{\textcolor[rgb]{0.68,0.00,0.00}{#1}}
\newcommand{\RegionMarkerTok}[1]{\textcolor[rgb]{0.00,0.23,0.31}{#1}}
\newcommand{\SpecialCharTok}[1]{\textcolor[rgb]{0.37,0.37,0.37}{#1}}
\newcommand{\SpecialStringTok}[1]{\textcolor[rgb]{0.13,0.47,0.30}{#1}}
\newcommand{\StringTok}[1]{\textcolor[rgb]{0.13,0.47,0.30}{#1}}
\newcommand{\VariableTok}[1]{\textcolor[rgb]{0.07,0.07,0.07}{#1}}
\newcommand{\VerbatimStringTok}[1]{\textcolor[rgb]{0.13,0.47,0.30}{#1}}
\newcommand{\WarningTok}[1]{\textcolor[rgb]{0.37,0.37,0.37}{\textit{#1}}}

\usepackage{longtable,booktabs,array}
\usepackage{calc} % for calculating minipage widths
% Correct order of tables after \paragraph or \subparagraph
\usepackage{etoolbox}
\makeatletter
\patchcmd\longtable{\par}{\if@noskipsec\mbox{}\fi\par}{}{}
\makeatother
% Allow footnotes in longtable head/foot
\IfFileExists{footnotehyper.sty}{\usepackage{footnotehyper}}{\usepackage{footnote}}
\makesavenoteenv{longtable}
\usepackage{graphicx}
\makeatletter
\newsavebox\pandoc@box
\newcommand*\pandocbounded[1]{% scales image to fit in text height/width
  \sbox\pandoc@box{#1}%
  \Gscale@div\@tempa{\textheight}{\dimexpr\ht\pandoc@box+\dp\pandoc@box\relax}%
  \Gscale@div\@tempb{\linewidth}{\wd\pandoc@box}%
  \ifdim\@tempb\p@<\@tempa\p@\let\@tempa\@tempb\fi% select the smaller of both
  \ifdim\@tempa\p@<\p@\scalebox{\@tempa}{\usebox\pandoc@box}%
  \else\usebox{\pandoc@box}%
  \fi%
}
% Set default figure placement to htbp
\def\fps@figure{htbp}
\makeatother





\setlength{\emergencystretch}{3em} % prevent overfull lines

\providecommand{\tightlist}{%
  \setlength{\itemsep}{0pt}\setlength{\parskip}{0pt}}



 


\KOMAoption{captions}{tableheading}
\makeatletter
\@ifpackageloaded{caption}{}{\usepackage{caption}}
\AtBeginDocument{%
\ifdefined\contentsname
  \renewcommand*\contentsname{Table of contents}
\else
  \newcommand\contentsname{Table of contents}
\fi
\ifdefined\listfigurename
  \renewcommand*\listfigurename{List of Figures}
\else
  \newcommand\listfigurename{List of Figures}
\fi
\ifdefined\listtablename
  \renewcommand*\listtablename{List of Tables}
\else
  \newcommand\listtablename{List of Tables}
\fi
\ifdefined\figurename
  \renewcommand*\figurename{Figure}
\else
  \newcommand\figurename{Figure}
\fi
\ifdefined\tablename
  \renewcommand*\tablename{Table}
\else
  \newcommand\tablename{Table}
\fi
}
\@ifpackageloaded{float}{}{\usepackage{float}}
\floatstyle{ruled}
\@ifundefined{c@chapter}{\newfloat{codelisting}{h}{lop}}{\newfloat{codelisting}{h}{lop}[chapter]}
\floatname{codelisting}{Listing}
\newcommand*\listoflistings{\listof{codelisting}{List of Listings}}
\makeatother
\makeatletter
\makeatother
\makeatletter
\@ifpackageloaded{caption}{}{\usepackage{caption}}
\@ifpackageloaded{subcaption}{}{\usepackage{subcaption}}
\makeatother
\usepackage{bookmark}
\IfFileExists{xurl.sty}{\usepackage{xurl}}{} % add URL line breaks if available
\urlstyle{same}
\hypersetup{
  pdftitle={Problem Set 10 Key},
  colorlinks=true,
  linkcolor={blue},
  filecolor={Maroon},
  citecolor={Blue},
  urlcolor={Blue},
  pdfcreator={LaTeX via pandoc}}


\title{Problem Set 10 Key}
\author{}
\date{}
\begin{document}
\maketitle


\begin{Shaded}
\begin{Highlighting}[]
\FunctionTok{library}\NormalTok{(tidyverse)}
\end{Highlighting}
\end{Shaded}

\begin{verbatim}
Warning: package 'ggplot2' was built under R version 4.5.1
\end{verbatim}

\begin{verbatim}
-- Attaching core tidyverse packages ------------------------ tidyverse 2.0.0 --
v dplyr     1.1.4     v readr     2.1.5
v forcats   1.0.0     v stringr   1.5.1
v ggplot2   4.0.0     v tibble    3.2.1
v lubridate 1.9.4     v tidyr     1.3.1
v purrr     1.0.4     
-- Conflicts ------------------------------------------ tidyverse_conflicts() --
x dplyr::filter() masks stats::filter()
x dplyr::lag()    masks stats::lag()
i Use the conflicted package (<http://conflicted.r-lib.org/>) to force all conflicts to become errors
\end{verbatim}

\section{Problem 1}\label{problem-1}

The dataset penguins preloaded in R includes data on the size and sex of
adult penguins in the Palmer Archipelago. Suppose we are interested in
testing for a correlation between body mass and flipper length.

\begin{Shaded}
\begin{Highlighting}[]
\FunctionTok{glimpse}\NormalTok{(penguins)}
\end{Highlighting}
\end{Shaded}

\begin{verbatim}
Rows: 344
Columns: 8
$ species     <fct> Adelie, Adelie, Adelie, Adelie, Adelie, Adelie, Adelie, Ad~
$ island      <fct> Torgersen, Torgersen, Torgersen, Torgersen, Torgersen, Tor~
$ bill_len    <dbl> 39.1, 39.5, 40.3, NA, 36.7, 39.3, 38.9, 39.2, 34.1, 42.0, ~
$ bill_dep    <dbl> 18.7, 17.4, 18.0, NA, 19.3, 20.6, 17.8, 19.6, 18.1, 20.2, ~
$ flipper_len <int> 181, 186, 195, NA, 193, 190, 181, 195, 193, 190, 186, 180,~
$ body_mass   <int> 3750, 3800, 3250, NA, 3450, 3650, 3625, 4675, 3475, 4250, ~
$ sex         <fct> male, female, female, NA, female, male, female, male, NA, ~
$ year        <int> 2007, 2007, 2007, 2007, 2007, 2007, 2007, 2007, 2007, 2007~
\end{verbatim}

\begin{enumerate}
\def\labelenumi{\alph{enumi}.}
\tightlist
\item
  Estimate the Spearman correlation coefficient for the sample. {[}2
  pts.{]}
\end{enumerate}

\begin{Shaded}
\begin{Highlighting}[]
\NormalTok{ctest }\OtherTok{\textless{}{-}} \FunctionTok{cor.test}\NormalTok{(penguins}\SpecialCharTok{$}\NormalTok{body\_mass,penguins}\SpecialCharTok{$}\NormalTok{flipper\_len,}\AttributeTok{method =} \StringTok{"spearman"}\NormalTok{)}
\end{Highlighting}
\end{Shaded}

\begin{verbatim}
Warning in cor.test.default(penguins$body_mass, penguins$flipper_len, method =
"spearman"): Cannot compute exact p-value with ties
\end{verbatim}

\begin{Shaded}
\begin{Highlighting}[]
\NormalTok{ctest}
\end{Highlighting}
\end{Shaded}

\begin{verbatim}

    Spearman's rank correlation rho

data:  penguins$body_mass and penguins$flipper_len
S = 1066875, p-value < 2.2e-16
alternative hypothesis: true rho is not equal to 0
sample estimates:
      rho 
0.8399741 
\end{verbatim}

The Spearman correlation coefficient is 0.8399741.

\begin{enumerate}
\def\labelenumi{\alph{enumi}.}
\setcounter{enumi}{1}
\tightlist
\item
  Perform a hypothesis test that tests whether there is a correlation
  between body mass and flipper length based on the Spearman correlation
  coefficient. What is the p-value? {[}2pts.{]}
\end{enumerate}

The p-value is \textless2.2e-16.

\begin{enumerate}
\def\labelenumi{\alph{enumi}.}
\setcounter{enumi}{2}
\tightlist
\item
  Is there evidence of a correlation between body mass and flipper
  length? Use a significance level of 0.01. {[}1 pt.{]}
\end{enumerate}

We reject the null hypothesis. There is sufficient evidence of a
correlation between body mass and flipper length.

\section{Problem 2}\label{problem-2}

The data set ``acupuncture.csv'' includes data from the control group of
a randomized controlled trial investigating the effect of acupuncture
therapy on headache severity. Each participant was identified using an
ID number in the id column. The headache severity was measured for each
participant before receiving the treatment and at a 3-month follow-up.
The column pk1 includes the baseline headache severity score of the
participants while the column pk2 includes the headache severity score
after 3 months.

\begin{Shaded}
\begin{Highlighting}[]
\NormalTok{acupuncture }\OtherTok{\textless{}{-}} \FunctionTok{read.csv}\NormalTok{(}\StringTok{"datasets/acupuncture.csv"}\NormalTok{)}
\FunctionTok{glimpse}\NormalTok{(acupuncture)}
\end{Highlighting}
\end{Shaded}

\begin{verbatim}
Rows: 173
Columns: 5
$ X     <int> 1, 2, 3, 4, 5, 6, 7, 8, 9, 10, 11, 12, 13, 14, 15, 16, 17, 18, 1~
$ id    <int> 112, 113, 114, 130, 131, 137, 138, 141, 149, 150, 161, 169, 184,~
$ group <int> 1, 1, 1, 1, 1, 1, 1, 1, 1, 1, 1, 1, 1, 1, 1, 1, 1, 1, 1, 1, 1, 1~
$ pk1   <dbl> 9.25, 42.50, 24.25, 21.75, 14.50, 11.75, 56.50, 15.50, 49.25, 9.~
$ pk2   <dbl> 4.75, 34.50, 16.25, 2.00, 20.00, 11.25, 70.50, 5.75, 9.75, 8.75,~
\end{verbatim}

\begin{enumerate}
\def\labelenumi{\alph{enumi}.}
\tightlist
\item
  Use a nonparametric test to test whether there is a difference between
  the average headache severity scores of the participants before
  receiving the treatment and the 3-month follow-up? Is the alternative
  hypothesis one-sided or two-sided? {[}3pt.{]}
\end{enumerate}

The alternative hypothesis is two-sided.

\(H_0:\) The median of the paired difference of headache scores is
zero./ The distribution of the paired difference of headache scores is
centered at zero.

\(H_a:\) The median of the paired difference of headache scores is not
zero./ The distribution of the paired difference of headache scores is
notcentered at zero.

\begin{enumerate}
\def\labelenumi{\alph{enumi}.}
\setcounter{enumi}{1}
\tightlist
\item
  What is the p-value? {[}1pt.{]}
\end{enumerate}

\begin{Shaded}
\begin{Highlighting}[]
\NormalTok{wtest }\OtherTok{\textless{}{-}} \FunctionTok{wilcox.test}\NormalTok{(}\AttributeTok{x=}\NormalTok{acupuncture}\SpecialCharTok{$}\NormalTok{pk1,}\AttributeTok{y=}\NormalTok{acupuncture}\SpecialCharTok{$}\NormalTok{pk2,}\AttributeTok{paired=}\NormalTok{T)}
\NormalTok{wtest}
\end{Highlighting}
\end{Shaded}

\begin{verbatim}

    Wilcoxon signed rank test with continuity correction

data:  acupuncture$pk1 and acupuncture$pk2
V = 12240, p-value = 4.799e-14
alternative hypothesis: true location shift is not equal to 0
\end{verbatim}

The p-value is \ensuremath{4.7992427\times 10^{-14}}.

\begin{enumerate}
\def\labelenumi{\alph{enumi}.}
\setcounter{enumi}{2}
\tightlist
\item
  At a significance level of 0.001, do we have sufficient evidence that
  the average headache severity score in this cohort differs at the
  3-month follow-up compared to the baseline? {[}1pt.{]}
\end{enumerate}

We reject the null hypothesis at 0.001 level of significance. We have
sufficient evidence that the average headache seveiry score differs at
the 3-month follow-up compared to the baseline.

\section{Problem 3}\label{problem-3}

The data set in hflashes.csv contains data from a 14-year cohort study
by Freeman et al (2001) that investigated the occurrence of hot flashes
in 375 participants.

\begin{Shaded}
\begin{Highlighting}[]
\NormalTok{hflash }\OtherTok{\textless{}{-}} \FunctionTok{read.csv}\NormalTok{(}\StringTok{"datasets/hflashes.csv"}\NormalTok{)}
\FunctionTok{glimpse}\NormalTok{(hflash)}
\end{Highlighting}
\end{Shaded}

\begin{verbatim}
Rows: 375
Columns: 14
$ pt       <int> 3, 6, 7, 8, 9, 10, 11, 12, 13, 14, 15, 16, 17, 19, 20, 23, 24~
$ ageg     <int> 2, 3, 1, 1, 2, 3, 2, 2, 2, 3, 2, 1, 1, 1, 1, 3, 2, 1, 1, 1, 2~
$ aagrp    <int> 0, 0, 0, 0, 0, 0, 0, 1, 0, 0, 1, 1, 0, 1, 1, 0, 0, 0, 0, 1, 0~
$ edu      <int> 1, 1, 1, 1, 1, 1, 1, 1, 1, 1, 1, 0, 1, 1, 1, 1, 0, 1, 1, 1, 1~
$ d1       <int> 0, 0, 0, 1, 0, 0, 0, 0, 1, 0, 1, 1, 1, 1, 0, 0, 0, 1, 0, 1, 0~
$ f1a      <int> 0, 0, 1, 0, 1, 0, 0, 1, 1, 0, 0, 1, 0, 0, 1, 1, 1, 0, 0, 0, 1~
$ pcs12    <dbl> 56.80537, 59.18338, 57.73952, 55.83575, 55.89324, NA, 55.5009~
$ hotflash <int> 0, 0, 0, 1, 0, 1, 0, 1, 1, 0, 1, 0, 0, 0, 0, 0, 0, 1, 0, 1, 0~
$ bmi30    <int> 0, 0, 0, 0, 0, 0, 1, 1, 0, 0, 1, 1, 0, 0, 0, 0, 0, 0, 0, 1, 0~
$ estra    <dbl> 106.710, 31.250, 13.410, 10.640, 24.060, 37.305, 26.320, 24.1~
$ fsh      <dbl> 3.005, 11.195, 14.545, 5.530, 9.780, 10.290, 7.960, 4.775, 7.~
$ lh       <dbl> 2.980, 5.760, 5.595, 2.260, 2.600, 3.395, 3.570, 2.095, 3.660~
$ testo    <dbl> 7.680, 11.930, 24.375, 8.280, 4.050, 8.275, 15.995, 12.340, 1~
$ dheas    <dbl> 61.225, 104.920, 117.450, 36.850, 11.165, 100.360, 76.780, 83~
\end{verbatim}

\begin{enumerate}
\def\labelenumi{\alph{enumi}.}
\tightlist
\item
  Use a non-parametric test to determine if there is a difference in the
  distribution of baseline estradiol measurements (Variable estra)
  between current (Variable f1a=1) and non-current smokers (Variable
  f1a=0). {[}3 pts.{]}
\end{enumerate}

We use the independent 2-sample Wilcoxon test, also known as the
Mann-Whitney U test. The hypotheses are:

\(H_0:\) The distribution/median of baseline estradiol measurements for
current and non-current smokers are the same. \(H_a:\) The
distribution/median of baseline estradiol measurements for current and
non-current smokers are not the same.

\begin{Shaded}
\begin{Highlighting}[]
\NormalTok{wtest }\OtherTok{\textless{}{-}} \FunctionTok{wilcox.test}\NormalTok{(estra}\SpecialCharTok{\textasciitilde{}}\NormalTok{f1a,}\AttributeTok{data=}\NormalTok{hflash)}
\NormalTok{wtest}
\end{Highlighting}
\end{Shaded}

\begin{verbatim}

    Wilcoxon rank sum test with continuity correction

data:  estra by f1a
W = 15390, p-value = 0.3415
alternative hypothesis: true location shift is not equal to 0
\end{verbatim}

\begin{enumerate}
\def\labelenumi{\alph{enumi}.}
\setcounter{enumi}{1}
\item
  What is the resulting p-value? {[}1 pt.{]} The p-value is 0.341461.
\item
  Comment on the strength/sufficiency of evidence against the null
  hypothesis. {[}1pt{]}
\end{enumerate}

We fail to reject the null hypothesis. There is insufficient evidence
that the distribution/median of baseline estradiol measurements for
current and non-current smokers are not the same.




\end{document}
